\section{Refinery and Fuels}
Crude oil (\SI{33.6}{\percent}), coke (\SI{27.2}{\percent}) and gas (\SI{23.9}{\percent}) are still the 3 main energy sources, followed by hydropower (\SI{6.8}{\percent}), nuclear power (\SI{4.4}{\percent}) and renewable energy sources (\SI{4}{\percent}).

Crude oil is the most important energy source due to its ease of extraction, wide range of applications and cheap transport.

\subsection{Kerogen}
Kerogen is an organic waxy material found in sedimentary rocks and formed from bacteria, decayed algae and wood.
It is the most common form of organic carbon in the earth.
Kerogen can be converted into synthetic oil at temperatures above \SI{500}{\celsius} or by hydration.
Canada (Athabasca Oil Sands) has the world's largest deposit of kerogen.
This deposit is close to the surface, making surface mining the primary method of extraction.
In this region mining of the oil sands dates back to 1967.
The worlds top reserve holders are China, United States, Argentina, Mexico, South Africa, Australia and Canada.

\subsubsection{Types of kerogen}
Kerogen can be divided into four different types.

\paragraph{Typ I (Sapropelic)}
Sapropelic kerogen contains the highest amount of oil and is therefore the most promising source.
It is derived form algea and its proteins and lipids.
It contains only few cyclic or aromatic structures.
Type I kerogen has high initial hydrogen-to-carbon ratios (H/C > 1.25) and low initial oxygen-to-carbon ratios (O/C < 0.15).

\paragraph{Type II (Planktonic)}
Planktonic kerogen is mostly based on marine organic materials.
It has a hydrogen-to-carbon ratio H/C smaller than 1.25 and initial oxygen-to-carbon ratio O/C of 0.03 to 0.18.

\paragraph{Typ III (Humic)}
Type III kerogens are characterized by low initial H/C ratios and high initial O/C ratios.
Type III kerogens are derived from terrestrial plant matter, specifically from precursor compounds including cellulose, lignin, terpenes and phenols.
Coal is an organic-rich sedimentary rock that is composed predominantly of this kerogen type.
On a mass basis, type III kerogens generate the lowest oil yield of principal kerogen types.
It has a hydrogen-to-carbon ratio H/C smaller than 1 and initial oxygen-to-carbon ratio O/C of 0.03 to 0.3.

\paragraph{Type IV (Residue)}
Type IV kerogen comprises mostly inert organic matter in the form of polycyclic aromatic hydrocarbons.
They have no potential to produce hydrocarbons
It has a hydrogen-to-carbon ratio H/C smaller than 0.5.

\subsection{Fracking}
Fracking is a method of creating, widening and stabilising cracks in the rock of a reservoir deep underground with the aim of increasing the permeability of the reservoir rocks.
This allows gases or liquids in them to flow more easily and steadily to the wellbore and be extracted.

In fracking, the fracking fluid is injected through a well under high pressure, typically several hundred bar, into the geological horizon from which it is to be extracted.
The fracking fluid is water, which is usually mixed with proppants, such as silica sand, thickeners and further additives.
Usually, several deviated wells (lateral wells) are first drilled into the reservoir by means of directional drilling, whereby the drill bit is guided parallel to the layers.
This means that the available borehole length in the reservoir is much greater, which generally increases the production yield.
Further additives are:

\begin{itemize}
    \item Gels (e.g. guar) to increase the viscosity of the fluid for better sand transport
    \item Foams (\ch{CO2} or \ch{N2}) for better transport and deposition on the proppant
    \item Acids (HCl, \ch{CH3COOH}, HCOOH, \ch{H3BO3}) for the dissolution of minerals
    \item Furthermore: Corrosion inhibitors, viscosity adjusters, biocides, fluid loss additives (preventing loss of fluid into neighboring rocks), friction minimizers
\end{itemize}

\subsection{Oil drilling}
Traditional methods of oil extraction have been the primary and secondary methods, which only exhaust between a quarter and half of a well’s oil reserves.
Such profligacy has been addressed by the development of a tertiary technique, more commonly known as enhanced oil recovery (EOR).

\subsubsection{Primary methods}
Primary oil recovery refers to the process of extracting oil either via the natural rise of hydrocarbons to the surface of the earth or via pump jacks and other artificial lift devices.
Since this technique only targets the oil, which is either susceptible to its release or accessible to the pump jack, this is very limited in its extraction potential.
Only around \SIrange{10}{20}{\percent} the well’s potential are recovered from the primary method.

\subsubsection{Secondary methods}
This method involves the injection of gas or water, which will displace the oil, force it to move from its resting place and bring it to the surface.
This is typically successful in targeting an additional \SI{30}{\percent} of the oil’s reserves.

\subsubsection{Tertiary methods (Enhanced Oil Recovery)}
Rather than simply trying to force the oil out of the ground, as did the previous two methods, enhanced oil recovery seeks to alter its properties to make it more conducive to extraction.
Tertiary methods target more than \SI{50}{\percent} of the oil’s reserves.
There are three main types of enhanced oil recovery

\paragraph{Thermal recovery}
This is the most prevalent type of EOR and works by heating the oil to reduce its viscosity and allowing easier flow to the surface.
This is most commonly achieved by introducing steam into the reservoir, which will work to heat the oil.

\paragraph{Gas injection} 
Either natural gas, nitrogen or carbon dioxide (increasingly the most popular option) are injected into the reservoir to mix with the oil, making it more viscous, whilst simultaneously pushing the oil to the surface (similar to secondary oil recovery).

\paragraph{Chemical injection}  
Polymers or surfactans are used to free trapped oil in the well.
This is done by lowering surface tension and increasing the efficiency of water-flooding.

\subsection{Crude Oils and Products}
\label{subsec:crude_oils_and_products}
Crude oil contains great variety of hydrocarbons (HCs).
There are three types of molecular structures are possible straight-chain, branched-chain and ring structures (Fig.~\ref{fig:hc_structs}).

\begin{figure}[H]
    \centering
    \includegraphics{Figures/HC_structures}
    \caption{Different possible hydrocarbon structures}
    \label{fig:hc_structs}
\end{figure}

Furthermore 
